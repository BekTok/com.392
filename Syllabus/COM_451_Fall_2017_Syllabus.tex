\documentclass[12pt,a4paper,oneside]{article}

\usepackage[margin=3cm]{geometry}

\usepackage{hyperref}
\hypersetup{
    pdftitle={COM 451, Parallel and Distributed Programming},%
    pdfauthor={Toksaitov Dmitrii Alexandrovich},%
    pdfsubject={Syllabus},%
    pdfkeywords={COM;}{451;}{syllabus;}{operating;}{systems},%
    colorlinks,%
    linkcolor=black,%
    citecolor=black,%
    filecolor=black,%
    urlcolor=black
}

\newcommand{\R}[1]{\uppercase\expandafter{\romannumeral #1\relax}}

\begin{document}

    \title{COM 451, Parallel and Distributed Programming}
    \author{
        American University of Central Asia\\
        Software Engineering Program
    }
    \date{}
    \maketitle

    \section{Course Information}

        \begin{description}
            \item[Course ID]\hfill\\
                COM 451, 3708
            \item[Course Repository]\hfill\\
                \url{https://github.com/auca/com.451}
            \item[Class Discussions]\hfill\\
                \url{https://piazza.com/class/j6tftykciix26}
            \item[Place]\hfill\\
                AUCA, laboratory G31
            \item[Time]\hfill\\
                Monday 14:10--15:25\\
                Monday 14:10--15:25
        \end{description}

    \section{Prerequisites}

        \begin{itemize}
            \item COM-324, Algorithm Analysis
            \item COM-341, Operating systems
        \end{itemize}

    \section{Contact Information}

        \begin{description}
            \item[Instructor]\hfill\\
                Toksaitov Dmitrii Alexandrovich\\
                \href{mailto:toksaitov_d@auca.kg}{toksaitov\_d@auca.kg}
            \item[Office]\hfill\\
                AUCA, room 315
            \item[Office Hours]\hfill\\
                Monday 15:25--17:00\\
                Tuesday 15:25--17:00\\
                Wednesday 10:00--17:00\\
                Thursday 15:25--17:00\\
                Friday 15:25--17:00
        \end{description}

    \section{Course Overview}

        The course introduces students to the topic of programming multi-core
        multi-processor systems with data-parallel facilities of various
        hardware on a single machine or in a distributed system connected to a
        high-performance network. The students will learn the most popular
        shared-memory parallel programming API such as Pthreads and a
        distributed memory programming API such as MPICH. Students will get a
        chance to work on two projects accelerating image processing tasks and
        astronomy simulations. Results of the projects will be tested together
        with students on high-performance parallel machines on the Amazon's
        cloud.

    \section{Topics Covered}

        \begin{itemize}
        \item Flynn's taxonomy
        \item Amdahl’s law
        \item CPU caches and locality
        \item CPU pipelines and branch prediction
        \item Hardware multithreading
        \item Shared and distributed memory systems
        \item NUMA and UMA architectures
        \item Data-parallelism with SIMD instructions
        \item General-purpose computing on graphics processing units
        \item Synchronization
        \begin{itemize}
          \item Spinlocks, barriers, mutexes, semaphores
          \item Conditional variables
          \item Race conditions
          \item Deadlocks
        \end{itemize}
        \end{itemize}

    \section{Examinations}

        Students will get midterm and final examinations in the form of two
        quizzes with multiple choice or open questions. Exam samples can be
        found on the official course repository.

    \section{Course Projects}

        Throughout the course, students will work on two major projects to
        accelerate programs performing image processing tasks and astronomy
        simulations.

        \subsection{Project \#1}

            In the first projects, students will be asked to accelerate a Sobel
            and Median filters in an image-processing program by utilizing all
            the cores of a test machine through the Pthread API. They may also
            accelerate their solutions even further by utilizing the parallel
            vector-processing facilities such as SIMD instructions of modern
            CPUs. For extra points, students may move the image processing
            algorithms to the GPU and compare the achieved speed up.

        \subsection{Project \#2}

            The second project requires accelerating the N-body simulation in a
            distributed environment with a high-speed interconnection between
            machines through the Message Passing Interface. Students will
            compete on how many planetary bodies can their accelerated systems
            handle compared to others.

    \section{Reading}

        An Introduction to Parallel Programming by Peter Pacheco (ISBN:
        978-0123742605)

        \subsection{Supplemental Reading}

            \begin{enumerate}
                \item Parallel Programming, 2nd Edition by Thomas Rauber and
                Gudula Rünger (ISBN: 978-3642048173)
                \item Computer Architecture: A Quantitative Approach, 5th
                Edition by David Patterson and John L. Hennessy (ISBN:
                978-0123838728)
            \end{enumerate}

    \section{Grading}

        \begin{itemize}
            \item Class participation (through Piazza) (5\%)
            \item Midterm (15\%)
            \item Final (20\%)
            \item Course projects (60\%)
        \end{itemize}

        \begin{itemize} \itemsep-10pt \parskip0pt \parsep0pt
            \item[--] 90\%--100\%: A\\
            \item[--] 80\%--89\%: A-\\
            \item[--] 70\%--79\%: B+\\
            \item[--] 65\%--69\%: B\\
            \item[--] 60\%--64\%: B-\\
            \item[--] 56\%--59\%: C+\\
            \item[--] 53\%--55\%: C\\
            \item[--] 50\%--52\%: C-\\
            \item[--] 46\%--49\%: D+\\
            \item[--] 43\%--45\%: D\\
            \item[--] 40\%--42\%: D-\\
            \item[--] Less than 39\%: F
        \end{itemize}

    \section{Rules}

        Students are required to follow the rules of conduct of the Software
        Engineering Department and American University of Central Asia.

        Team work is NOT encouraged. The same blocks of code or similar
        structural pieces in separate works will be considered as academic
        dishonesty and all parties will get zero for the task.

\end{document}

