\documentclass[12pt,a4paper,oneside]{article}

\usepackage[margin=3cm]{geometry}

\usepackage{hyperref}
\hypersetup{
    pdftitle={COM 451, Parallel and Distributed Programming},%
    pdfauthor={Toksaitov Dmitrii Alexandrovich},%
    pdfsubject={Syllabus},%
    pdfkeywords={COM;}{451;}{syllabus;}{operating;}{systems},%
    colorlinks,%
    linkcolor=black,%
    citecolor=black,%
    filecolor=black,%
    urlcolor=black
}

\newcommand{\R}[1]{\uppercase\expandafter{\romannumeral #1\relax}}

\begin{document}

    \title{COM 451, Parallel and Distributed Programming}
    \author{
        American University of Central Asia\\
        Department of Software Engineering
    }
    \date{}
    \maketitle

    \section{Course Information}

        \begin{description}
            \item[Course ID]\hfill\\
                COM 451, 3708
            \item[Course Repository]\hfill\\
                \url{https://github.com/auca/com.451}
            \item[Class Discussions]\hfill\\
                \url{https://piazza.com/class/j6tftykciix26}
            \item[Place]\hfill\\
                AUCA, laboratory G31
            \item[Time]\hfill\\
                Monday 14:10--15:25\\
                Monday 14:10--15:25
        \end{description}

    \section{Prerequisites}

        \begin{itemize}
            \item COM-324, Algorithm Analysis
            \item COM-341, Operating systems
        \end{itemize}

    \section{Contact Information}

        \begin{description}
            \item[Instructor]\hfill\\
                Toksaitov Dmitrii Alexandrovich\\
                \href{mailto:toksaitov_d@auca.kg}{toksaitov\_d@auca.kg}
            \item[Office]\hfill\\
                AUCA, room 315
            \item[Office Hours]\hfill\\
                Monday 15:25--17:00\\
                Tuesday 15:25--17:00\\
                Wednesday 10:00--17:00\\
                Thursday 15:25--17:00\\
                Friday 15:25--17:00
        \end{description}

    \section{Course Overview}

        The course introduces students to the topic of programming multi-core
        multi-processor systems with data-parallel facilities of various
        hardware on a single machine or in a distributed system connected to a
        high-performance network.

    \section{Topics Covered}

        \begin{itemize}
            \item Threading with \textit{pthreads}
            \item Synchronization
            \item Data-parallelism with SIMD instructions
            \item GPGU programming with \textit{OpenCL} and \textit{CUDA}
            \item Distributed programming with \textit{OpenMPI}
        \end{itemize}

    \section{Examinations}

        Students will get midterm and final examinations in the form of two
        quizzes with multiple choice questions.

    \section{Course Projects}

        Throughout the course, students will have to work on three major
        projects to accelerate programs for image processing, astronomy
        simulation, and rendering through ray marching.

    \section{Reading}

        An Introduction to Parallel Programming by by Peter Pacheco (ISBN:
        978-0123742605)

    \section{Grading}

        \begin{itemize}
            \item Class participation (through Piazza) (5\%)
            \item Midterm (15\%)
            \item Final (20\%)
            \item Course projects (60\%)
        \end{itemize}

        \begin{itemize} \itemsep-10pt \parskip0pt \parsep0pt
            \item[--] 90\%--100\%: A\\
            \item[--] 80\%--89\%: A-\\
            \item[--] 70\%--79\%: B+\\
            \item[--] 65\%--69\%: B\\
            \item[--] 60\%--64\%: B-\\
            \item[--] 56\%--59\%: C+\\
            \item[--] 53\%--55\%: C\\
            \item[--] 50\%--52\%: C-\\
            \item[--] 46\%--49\%: D+\\
            \item[--] 43\%--45\%: D\\
            \item[--] 40\%--42\%: D-\\
            \item[--] Less than 39\%: F
        \end{itemize}

    \section{Rules}

        Students are required to follow the rules of conduct of the Software
        Engineering Department and American University of Central Asia.

        Team work is NOT encouraged. The same blocks of code or similar
        structural pieces in separate works will be considered as academic
        dishonesty and all parties will get zero for the task.

\end{document}

